\documentclass[notes.tex]{subfiles}

\begin{document}

Redes complexas, como definido por \citeonline{metz2007redes}, são grafos com uma topologia não trivial.
Isso é, são grafos onde parte ou toda a informação de interesse está contida não nos vértices e arestas individualmente, mas em propriedade do conjunto de vértices e arestas.

Como apontado por \citeonline{girvan2002community}, um dos sistemas do mundo real que se pode modelar em uma rede complexa é o conjunto de relações sociais.
Esse sistema poderia ser modelado com cada individuo sendo representado por um vértice, e as relações sendo representadas pelas arestas ligando dois indivíduos.
Num modelo assim, se é observada na topologia do grafo a existência de um sub grafo completo, denominado clique \cite{fortunato2010community}, isso representa não só que existe um grupo onde todos se conhecem, mas de que esse grupo pode apresentar características próprias e relevantes.

\citeonline{girvan2002community} também aponta que outros sistemas, como cadeias alimentares, cadeias de metabolização, redes de transmissão elétrica e redes de computadores podem ser representadas como redes complexas.
Em cada sistema do mundo real que se modela como uma rede complexa, diferentes propriedades emergem, e elas apresentam diferentes significados sobre o grafo observado.
E.g. para a propriedade de comunidades, as características que definem o que é uma comunidade variam de acordo com o contexto que se está trabalhando \cite{da2020comparative}, como será discutido mais a frente.

No estudo de redes complexas, é relevante o entendimento de algumas terminologias que serão utilizadas mais à frente.
É definido por \citeonline{fortunato2010community} o conceito de \emph{clique}, um clique se refere à um conjunto de vértices que forma um sub grafo completo.
Definições ligeiramente distintas podem ser utilizadas, como apontado pelo próprio \citeonline{fortunato2010community}, dependendo do contexto, por exemplo é apresentada a definição de um \emph{n-clique}, como sendo um subgrafo onde todos os vértices estão a \emph{n} ou menos arestas de distância.
Ou ainda um \emph{k-clique chain}, sendo a união de dois cliques com \emph{k} ou mais vértices que compartilham um ou mais desses vértices.

Outra terminologia relevante é o conceito de partição, definido por \citeonline{fortunato2010community} como sendo a divisão dos vértices de um grafo em conjuntos distintos de forma que todos os vértices pertençam a exatamente uma partição.
Essas duas definições são relevantes na forma como elas interagem com a definição de comunidade.
Em alguns contextos uma comunidade pode ser definida como sendo necessariamente um \emph{clique} \cite{fortunato2010community}, ou em outros contextos a divisão das comunidades no grafo deve de representar uma partição \cite{da2020comparative, fortunato2010community}

Mais uma definição definição relevante pode ser encontrada no trabalho de \citeonline{fortunato2010community}.
O autor define triângulo como sendo um conjunto de três vértices conectados entre si, i.e. um \emph{3-clique}.
E a contra parte dessa definição, o coeficiente de aglomeração, definido como a proporção em que as triplas conexas de um grafo são triângulos.
Isso é, para um coeficiente de aglomeração igual a meio, isso representa que para quaisquer três vértices que estejam conectados entre si, existe cinquenta por cento de chance de esses vértices formarem um triângulo.

Por fim, no estudo de redes complexas, existe também a definição de um grafo aleatório.
Um grafo aleatório é produzido a partir de um grafo qualquer, mantendo a mesma quantidade de vértices e mantendo o mesmo grau, mas considerando que os vértices estão conetados baseados em uma probabilidade uniformemente distribuída.
Um grafo aleatório é geralmente utilizado para comparação, servindo como exemplo nulo, no sentido de que ele não apresenta características topológicas relevantes \cite{fortunato2010community}.

\section{Objetivos}

    asdf
\subsection{Objetivos 23}

    asdf

\end{document}
