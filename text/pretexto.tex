\documentclass[notes.tex]{subfiles}

% ---
% Redefinição de comandos para adequar ao formato requerido
% ---
\makeatletter
\renewcommand{\imprimircapa}{

    \DoubleSpacing

    \begin{centering}

        {\bfseries \MakeUppercase{\imprimirinstituicao}\par}
    
        \vspace*{\fill} \vspace*{\fill} \vspace*{\fill} \vspace*{\fill}
        \vspace*{\fill} \vspace*{\fill} \vspace*{\fill}
    
        {\bfseries\LARGE\MakeUppercase{\imprimirtitulo}}\\
    
        \vspace*{\fill}

        \begin{flushright}
            \bfseries \MakeUppercase{\imprimirautor}
        \end{flushright}   

        \vspace*{\fill} \vspace*{\fill} \vspace*{\fill} \vspace*{\fill}
        \vspace*{\fill} \vspace*{\fill}

        \SingleSpace

        {\bfseries \MakeUppercase{\imprimirlocal} \\ \MakeUppercase{\imprimirdata} }
        \vspace*{\fill}

    \end{centering}
    \pdfbookmark[0]{Capa}{}
    \pagebreak
}
\makeatother

% ---

\makeatletter
\renewcommand{\imprimirfolhaderosto}{

    \SingleSpace
    \begin{centering}

        {\bfseries \MakeUppercase{\imprimirautor}\par}
    
        \vspace*{\fill} \vspace*{\fill}
    
        {\bfseries\LARGE\MakeUppercase{\imprimirtitulo}}\\
    
        \vspace*{\fill}

        \begin{flushright}
            \parbox{0.5\textwidth}{%
                Trabalho de Conclusão de Curso apresentado ao curso de graduação em Ciências da Computação no Centro de de Ciências Exatas e Naturais da Universidade Regional de Blumenau como requisito parcial para a obtenção de grau de Bacharel em Ciências da Computação.
            }
        \end{flushright}   

        \begin{flushright}
            \parbox{\textwidth*3/5}{%
                Professor \imprimirorientador, Mestre - Orientador
            }
        \end{flushright}   

        \vspace*{\fill} \vspace*{\fill} \vspace*{\fill}
        \vspace*{\fill} \vspace*{\fill} \vspace*{\fill}

        {\bfseries \MakeUppercase{\imprimirlocal} \\ \MakeUppercase{\imprimirdata} }

    \end{centering}
    \pagebreak
}
\makeatother

% ---

\makeatletter
\newcommand{\imprimirfolhadeassinaturas}{
    \SingleSpace
    \begin{centering}
        \vspace*{\fill} 
        {\bfseries\LARGE\MakeUppercase{Folha de assinaturas}} 
        %\includepdf{folhadeaprovacao_final.pdf}
        \vspace*{\fill}
    \end{centering}
    \pagebreak
}
\makeatother

% ---

\begin{document}
% página de titulo principal (obrigatório)
\imprimircapa

\imprimirfolhaderosto

\imprimirfolhadeassinaturas

\begin{dedicatoria}
    \vspace*{\fill}
    \hfill
    \parbox{0.5\textwidth}{%
        Dedico esse trabalho a minha noiva, cuja paciência em me ouvir falar desse trabalho tornou-o possível.
    }
    \vspace*{\fill}
\end{dedicatoria}
\pagebreak

\DoubleSpacing
\begin{agradecimentos}
    A meu padrinho, Maiko Rafael Spiess, pelo sempre presente incentivo ao estudo.

    Ao meu orientador, Aurélio Faustino Hoppe, por acreditar na conclusão desse trabalho.

    A minha família, por todos os anos de apoio que foram necessários para chegar até aqui.

    Aos amigos que fiz no percurso do bacharelado, pelo apoio recebido.

    Aos professores do Departamento de Sistemas e Computação da Universidade Regional de Blumenau por suas contribuições durante os semestres letivos.
\end{agradecimentos}
\pagebreak

\SingleSpace
\begin{epigrafe}
    \vspace*{\fill}

    \hfill
    \parbox{0.5\textwidth}{%
    ``Se eu vi mais longe, foi por estar sobre ombros de gigantes.''
    }

    \begin{flushright}
    Isaac Newton
    \end{flushright}
    \vspace*{\fill}
\end{epigrafe}

% resumo em português
\DoubleSpacing
\begin{resumoumacoluna}
\bigskip

\lipsum[1]

 \vspace{\onelineskip}

 \noindent
 \textbf{Palavras-chave}: Redes complexas. Geração de redes complexas. Comunidades. Comunidades sobrepostas. Comunidades hierárquicas.
\end{resumoumacoluna}


% resumo em inglês
\renewcommand{\resumoname}{Abstract}
\begin{resumoumacoluna}
 \begin{otherlanguage*}{english}
\bigskip

   \lipsum[2]

   \vspace{\onelineskip}

   \noindent
   \textbf{Keywords}: Complex networks. Complex networks generation. Communities. Overlapping communities. Hierarchical communities
 \end{otherlanguage*}
\end{resumoumacoluna}

\pdfbookmark[0]{\listfigurename}{lof}
\listoffigures*
\pagebreak

\pdfbookmark[0]{\listofquadrosname}{loq}
\listofquadros*
\pagebreak

\pdfbookmark[0]{\listtablename}{lot}
\listoftables*
\pagebreak

\begin{siglas}
\item[IE] -- Internet Explorer
\item[QE] -- Qternet Explorer
\end{siglas}
\pagebreak

\tableofcontents* 

\end{document}
