\documentclass[notes.tex]{subfiles}
\begin{document}

\chapter{Resultados experimentais}

Foi realizada uma implementação do algoritmo descrito na linguagem python 3.8.
A partir desse, foram executados os testes para avaliar a presença das propriedades desejadas no modelo.
Para os testes iniciais, foram utilizados valores descritos no \autoref{qua:base_params}.

\begin{quadro}[htbp]
    \centering
    \caption{Parâmetros básicos}
    \label{qua:base_params}
    \begin{tblr}{|r|r|r|r|r|r|r|r|} \hline
        \SetCell{c} $N$ & \SetCell{c} $E_\text{wth}^\text{max}$ & \SetCell{c} $E_\text{btw}^\text{max}$ & \SetCell{c} $MTE$ & \SetCell{c} $\A$ & \SetCell{c} $K$ & \SetCell{c} $\theta$ & \SetCell{c} $\text{NbRep}$ \\ \hline
         1000 & 45 & 3 & 30000 & (1, 1) & (6, 2) & $\sfrac{1}{10}$ & 20 \\ \hline
    \end{tblr}
    \fonte{elaborado pelo autor}
\end{quadro}

\section{Comunidades}

A presença de comunidades como uma propriedade de redes complexas está tipicamente atrelada a ideia de haverem mais arestas internas a uma comunidade do que arestas que atravessem a fronteira entre duas comunidades.
Esse entendimento é solidificado na função de modularidade $Q$ de \citeonline{girvan2002community}, conforme descrito na equação \ref{eq:mod}.
\citeonline{shen2009detect} descreve uma adaptação, modularidade estendida $EQ$, para a aplicação em coberturas (em oposição a partições), conforme equação \ref{eq:mod_e}.
Essa aplicação em coberturas destaca a possibilidade de comunidades sobrepostas.
A aplicação de $EQ$ recursivamente, de forma a considerar o subgrafo com os membros da comunidade dividido entre as comunidades que a compõe descreve a propriedade de comunidades hierárquicas.

\begin{equation}
O_i = \left[\sum_{c \in C_n| L_{c} = i-1}V_c\right] - \left[\sum_{c \in C_n| L_{c} = i}V_c\right]
\end{equation}

A presença de sobreposição nas comunidades pode ser medida como um função $O_i$ que calcula a quantidade de vértices presentes em mais de uma comunidade de nível $i$.
O valor é dado pela diferença da quantidade de vértices em comunidades de nível $i$ e a quantidade de vértices em comunidades de nível $i+1$.
O efeito disso é que vértices que estejam em duas comunidades distintas de nível $i+1$ ou menores, serão contados duas vezes, mas os que pertencem a duas comunidades de nível $i$ ou menores serão compensadas.
Por fim, a soma de $O_i$ para todos os níveis será anotada como  $O$.

Observa-se na \autoref{tab:mod_base_params} os valores de $EQ$ para 10 execuções do modelo com os parâmetros básicos ($B_0$ a $B_9$).
Para as comunidades de primeiro nível, isso é, as que são compostas por sub comunidades, a função gera uma média de aproximadamente 0,8 o que indica uma estrutura de comunidade muito expressiva.
No segundo nível, o valor médio foi de aproximadamente 0,6 indicando comunidades menos expressivas mas ainda presentes.

\begin{table}[htbp]
    \centering
    \caption{Modularidade com os parâmetros básicos}
    \label{tab:mod_base_params}
    \begin{tblr}{l|r|r|r|r} \hline
        \SetCell{c} Execução & \SetCell{c} $EQ_1$ & \SetCell{c} $EQ_2$  & \SetCell{c} $O_1$ & \SetCell{c} $O_2$\\ \hline
        $B_0$ & 0,79749 & 0,59244 & 18 & 240 \\ \hline
        $B_1$ & 0,79729 & 0,60191 & 20 & 227 \\ \hline
        $B_2$ & 0,80668 & 0,61159 & 10 & 224 \\ \hline
        $B_3$ & 0,79498 & 0,64428 & 19 & 188 \\ \hline
        $B_4$ & 0,79230 & 0,61631 & 23 & 193 \\ \hline
        $B_5$ & 0,78668 & 0,56342 & 25 & 254 \\ \hline
        $B_6$ & 0,80528 & 0,62212 & 14 & 209 \\ \hline
        $B_7$ & 0,81357 & 0,59754 & 12 & 228 \\ \hline
        $B_8$ & 0,81050 & 0,58689 & 11 & 238 \\ \hline
        $B_9$ & 0,78874 & 0,57122 & 20 & 252 \\ \hline
        Média & 0,79935 & 0,60077 & 17 & 225\\ \hline
    \end{tblr}
    \fonte{elaborado pelo autor}
\end{table}

A \autoref{tab:mod_base_params} indica também os valores de sobreposição entre comunidades de primeiro ($O_1$) e segundo nível ($O_2$).
Os valores indicam quem em média, dos mil vértices presentes em um grafo como o observado, dezessete estarão em duas comunidades de primeiro nível e em média duzentos e vinte e cinto vértices estarão em duas comunidades de segundo nível.
O que esses valores indicam é que até vinte e quatro por cento dos vértices, para essa parametrização, estariam distribuídos em mais de uma comunidade.

Naturalmente, a escolha de valores distintos para os parâmetros pode ser feita para reforçar as características desejáveis.
Limitando o valor de $MTE$ e considerando uma estrutura de comunidades gerada com  $K = (9, 2, 2, 2)$, obteve-se os dados distritos na \autoref{tab:mod_pro_params}.
Observa-se que a redução do número mínimo foi feita para evitar a descaracterização das comunidades folha, considerando que o valor de $K$ gera 72 dessas.

\begin{table}[htbp]
    \centering
    \caption{Modularidade com $K = (9, 2, 2, 2)$}
    \label{tab:mod_pro_params}
    \begin{tblr}{l|r|r|r|r|r} \hline
        \SetCell{c} Execução & \SetCell{c} $EQ_1$ & \SetCell{c} $EQ_2$ & \SetCell{c} $EQ_3$ & \SetCell{c} $EQ_4$ & \SetCell{c} $O$ \\ \hline
        $M_0$ & 0,87488 & 0,92136 & 0,93030 & 0,90631 & 0 \\ \hline
        $M_1$ & 0,87855 & 0,92355 & 0,93257 & 0,90777 & 0 \\ \hline
        $M_2$ & 0,87783 & 0,92417 & 0,93343 & 0,90915 & 0 \\ \hline
        $M_3$ & 0,87769 & 0,92300 & 0,93276 & 0,90888 & 0 \\ \hline
        $M_4$ & 0,86678 & 0,91529 & 0,92477 & 0,90950 & 0 \\ \hline
        Média & 0,87515 & 0,92147 & 0,93077 & 0,90832 & 0 \\ \hline
    \end{tblr}
    \fonte{elaborado pelo autor}
\end{table}

O incremento nos valores obtidos é atribuível à existência de mais comunidades para as quais distribuir os vértices, não apenas em profundidade mas em diâmetro também.
Descartando-se a possibilidade de ser a alteração do valor de $MTE$ ao realizar testes com este zerado mas mantendo o $K$ padrão.
Em compensação, essa parametrização faz com que a escolha de comunidades folha se torne muito mais provável, anulando a possibilidade de comunidades sobrepostas.
Esse comportamento ocorre por no processo de escolha das comunidades a ordenação das comunidades se dá por escolher quais comunidades tem os representantes mais próximos, e as comunidades folha por serem em maior quantidade são muito mais prováveis de serem escolhidas.

Outros testes descritos na \autoref{tab:mod_var_params}, onde com $MTE = 0$ realizou-se uma exploração dos valores de $EQ$ para diversos valores em $K$.
Note-se o caso $M_9$ como contra exemplo do impacto de $K$ na modularidade, bem como o exemplo $M_6$, que nas comunidades de primeiro teme $EQ = 0,5$.

\begin{table}[htbp]
    \centering
    \caption{Modularidade com variação de $K$}
    \label{tab:mod_var_params}
    \begin{tblr}{l|r|r} \hline
        \SetCell{c} Execução & $K$ & \SetCell{c} Média de $EQ$ \\ \hline
        % (0.85619 + 0.93076 + 0.79259) /3 = 0.859847
        $M_5$ & $(5, 5, 5)$ & 0,859847 \\ \hline 
        % (0.91485 + 0.86146 + 0.49883) /3 = 0.75838
        $M_6$ & $(2, 4, 8)$ & 0,75838 \\ \hline
        % (0.91605 + 0.9208 + 0.86662) /3 = 0.901157
        $M_7$ & $(8, 2, 4)$ & 0,901157 \\ \hline
        $M_8$ & $(100)$ & 0,87249 \\ \hline
        $M_9$ & $(3)$ & 0,34691 \\ \hline
    \end{tblr}
    \fonte{elaborado pelo autor}
\end{table}

Alternativamente, é possível modificar significativamente os valores obtidos para a modularidade com os parâmetros $E_\text{btw}^\text{max}$ e $E_\text{wth}^\text{max}$.
Na execução $M_{10}$ utilizou-se os demais parâmetros nos valores padrão, mas com $E_\text{btw}^\text{max}=45$ e $K = (2, 2)$, para um $EQ$ de 0,4.

\section{Homofilia e Homogeneidade}

Homogeneidade é a principal propriedade das comunidades definidas por semelhança de vértices, ela está intimamente ligada à propriedade de homofilia.
Assumindo a distância euclideana como função de semelhança, a forma natural de medir a homogeneidade das comunidades descritas por uma cobertura é a inercia e a razão da inercia entre diferentes níveis de comunidades.

\subsection{Inercia}

Inercia, definida como a soma dos quadrados das distancias de cada membro de um conjunto com a média do conjunto, sumariza quão semelhantes ou dissemelhantes os membros são.
A razão entre a inercia em diferentes níveis é facilmente extrapolada do conceito de razão de inercia introduzido por \citeonline{largeron2015generating}.
Assumindo que as comunidades de uma rede complexa sejam homogêneas e hierárquicas, é natural que as comunidades mais específicas (sub comunidades), tenham menos diversidade que as comunidade mais generalistas.
Tendo em vista que a inercia calculada com uma cobertura deve levar em conta que vértices serão contados múltiplas vezes se estiverem presentes em múltiplas comunidades, a inercia para um nível $l$ segue descrita na equação \ref{eq:inercia_l}.
Considerando que $C_l$ é o conjunto de comunidades de nível  $l$.

\begin{equation}\label{eq:inercia_l}
    I_l = \sum_{c \in C_l}\frac{\sum_{v \in c} |v-g_c|^2}{|c|}
\end{equation}

$I_l$ apresenta uma forma de comparação entre diferentes coberturas, de forma a identificar qual seria mais homogênea dado um mesmo conjunto do vértices.
A extensão do conceito de inercia para a razão entre a inercia aplicada a dois níveis distintos se dá para isolar o fator de escala.
Isso é, $\sfrac{I_l}{I_0}$ descreve não apenas quão homogêneo são os agrupamentos estabelecidos, mas o quão mais homogêneos eles são em relação à população em geral.
Note-se que $I_0$ nesse contexto se refere à comunidade $C_n$, isso é, a comunidade que contém todos os vértices.

\subsection{Distância esperada}

A homofilia, como propriedade de redes complexas onde os vértices tendem a estarem adjacentes à vértices semelhantes.
Em um sistema com atributos discretos, é facilmente estabelecido uma função $\Delta$  que descreve o quão mais semelhantes os vértices adjacentes são do que seria esperado em um modelo nulo.
O modelo nulo trivial neste caso é a consideração de vértices conectados com igual probabilidade, mantendo-se a mesma quantidade de vértices e arestas, conforme descrito por \citeonline{easley2010networks}.

A adaptação para um modelo de um espaço contínuo de múltiplas dimensões se dá calculando uma aproximação da distância média entre cada par de vértices, essa média, $H_e$, é o valor que seria esperado se o grafo não fosse homofílico.
O valor real é calculado como a média da distância dos pontos adjacentes, $H_r$.

\subsection{Resultados}

Na \autoref{tab:iner_base_params} estão discriminados os valores de inércia médios para as comunidades organizadas por nível.
A leitura que se pode fazer desses dados, em especial das médias, é: em média uma comunidade de primeiro nível tem 38\% da diversidade que se encontra na população geral; e em média uma comunidade de segundo nível (sub comunidade) tem 33\% da diversidade presente na população em geral.
Por fim, em média a distância entre dos vértices adjacentes é metade do que seria esperado sem a homofilia, indicando a presença dessa propriedade.

\begin{table}[htbp]
    \centering
    \caption{Homogeneidade e homofilia com os parâmetros básicos}
    \label{tab:iner_base_params}
    \begin{tblr}{l|r|r|r|r} \hline
        \SetCell{c} Execução & \SetCell{c} $\sfrac{I_1}{I_0}$ & \SetCell{c} $\sfrac{I_2}{I_0}$ & \SetCell{c} $\sfrac{I_1+I_2}{2I_0}$ & \SetCell{c} $\sfrac{H_r}{H_e}$
        \\ \hline
$B_0$ & 0,37838 & 0,32957 & 0,353975 & 0,51867 \\ \hline
$B_1$ & 0,37367 & 0,34085 & 0,357260 & 0,54287 \\ \hline
$B_2$ & 0,34342 & 0,28694 & 0,315180 & 0,48891 \\ \hline
$B_3$ & 0,39089 & 0,35864 & 0,374765 & 0,54219 \\ \hline
$B_4$ & 0,39894 & 0,35848 & 0,378710 & 0,55024 \\ \hline
$B_5$ & 0,33398 & 0,27633 & 0,305155 & 0,48106 \\ \hline
$B_6$ & 0,45564 & 0,38316 & 0,419400 & 0,55749 \\ \hline
$B_7$ & 0,41636 & 0,36857 & 0,392465 & 0,57048 \\ \hline
$B_8$ & 0,32854 & 0,28121 & 0,304875 & 0,49867 \\ \hline
$B_9$ & 0,37391 & 0,32310 & 0,348505 & 0,53405 \\ \hline
Média & 0,37937 & 0,33069 & 0,355030 & 0,52846 \\ \hline
    \end{tblr}
    \fonte{elaborado pelo autor}
\end{table}

Naturalmente, existe uma influencia significativa que a quantidade de comunidades tem neste valor.
Observe-se na \autoref{tab:iner_pro_params}, com mais comunidades para distribuir os vértices, as comunidades ficam significativamente mais homogêneas.
Como efeito disso, a distância entre vértices conectados é bastante reduzida.

\begin{table}[htbp]
    \centering
    \caption{Homofilia e Homogeneidade com $K = (9, 2, 2, 2)$}
    \label{tab:iner_pro_params}
    \begin{tblr}{l|r|r|r|r|r} \hline
        \SetCell{c} Execução & \SetCell{c} $\sfrac{I_1}{I_0}$ & \SetCell{c} $\sfrac{I_2}{I_0}$ & \SetCell{c} $\sfrac{I_3}{I_0}$ & \SetCell{c} $\sfrac{I_4}{I_0}$ & \SetCell{c} $\sfrac{H_r}{H_e}$ \\ \hline
        $M_0$ & 0,18520 & 0,10856 & 0,05518 & 0,03054 & 0,17945 \\ \hline
        $M_1$ & 0,18758 & 0,11560 & 0,06015 & 0,03193 & 0,18039 \\ \hline
        $M_2$ & 0,18446 & 0,11812 & 0,05993 & 0,03546 & 0,18521 \\ \hline
        $M_3$ & 0,18714 & 0,11471 & 0,06123 & 0,03290 & 0,18356 \\ \hline
        $M_4$ & 0,18125 & 0,11156 & 0,05556 & 0,03016 & 0,17048 \\ \hline
        Média & 0,18513 & 0,11371 & 0,05841 & 0,03220 & 0,17982 \\ \hline
    \end{tblr}
\fonte{elaborado pelo autor}
\end{table}

Também de forma natural, existe uma relação entre o parâmetro $\theta$ e a inercia.
É apontado na tabela \autoref{tab:iner_var_params} a variação da inercia conforme se altera $\theta$.
Duas características dos dados descritos são relevantes, primeiramente com $\theta = 1$ se a ortogonalidade das comunidades não estivesse tendo efeito, seria de se esperar comunidades onde a diversidade interna se aproximaria da diversidade da população geral, mas não é isso o que se observa.
Isso é, em um modelo nulo, onde não existisse correlação entre os atributos do vértice e a comunidade em que ele participa, seria esperado uma razão de inercia que tendesse as um.

\begin{table}[htbp]
    \centering
    \caption{Homofilia e Homogeneidade com $\theta$ variável}
    \label{tab:iner_var_params}
    \begin{tblr}{l|r|r|r|r} \hline
        \SetCell{c} Execução & \SetCell{c} $\theta$ & \SetCell{c} $\sfrac{I_1}{I_0}$ & \SetCell{c} $\sfrac{I_2}{I_0}$ &
        \SetCell{c} Média
        \\ \hline
        $I_0$ & \multirow{3}{*}{1.00} & 0,74099 & 0,72866 & \multirow{3}{*}{0.72601} \\ \hline
        $I_1$ &                       & 0,69068 & 0,67882 &                          \\ \hline
        $I_2$ &                       & 0,77254 & 0,74436 &                          \\ \hline
        $I_3$ & \multirow{3}{*}{0.66} & 0,42301 & 0,38883 & \multirow{3}{*}{0.38678} \\ \hline
        $I_4$ &                       & 0,44184 & 0,37531 &                          \\ \hline
        $I_5$ &                       & 0,37331 & 0,31835 &                          \\ \hline
        $I_6$ & \multirow{3}{*}{0.33} & 0,43381 & 0,38709 & \multirow{3}{*}{0.39471} \\ \hline
        $I_7$ &                       & 0,36789 & 0,30745 &                          \\ \hline
        $I_8$ &                       & 0,46135 & 0,41065 &                          \\ \hline
    \end{tblr}
\fonte{elaborado pelo autor}
\end{table}

O segundo ponto a ser destacado é que $\theta$ não é um controle direto sobre a inercia, nos casos com  $\theta=0.66$, $\theta=0.33$ e $\theta=0.1$ (\autoref{tab:iner_base_params}), a razão se manteve em um mesmo patamar, entre 30\% e 40\%.
Existem outros fatores que afetam diretamente a homogeneidade das comunidades, e que são mais dificilmente controlados por parâmetros.

\end{document}
