\documentclass[notes.tex]{subfiles}
\begin{document}

\chapter{Resultados experimentais}

Foi realizada uma implementação do algoritmo descrito na linguagem python 3.8.
A partir desse, foram executados os testes para avaliar a presença das propriedades desejadas no modelo.
Para os testes iniciais, foram utilizados valores descritos no \autoref{qua:base_params}.

\begin{quadro}[htbp]
    \centering
    \caption{Parâmetros básicos}
    \label{qua:base_params}
    \begin{tblr}{|r|r|r|r|r|r|r|r|} \hline
        \SetCell{c} $N$ & \SetCell{c} $E_\text{wth}^\text{max}$ & \SetCell{c} $E_\text{btw}^\text{max}$ & \SetCell{c} $MTE$ & \SetCell{c} $\A$ & \SetCell{c} $K$ & \SetCell{c} $\theta$ & \SetCell{c} $\text{NbRep}$ \\ \hline
         1000 & 45 & 3 & 7000 & (1, 1) & (6, 2) & $\sfrac{1}{10}$ & 20 \\ \hline
    \end{tblr}
    \fonte{elaborado pelo autor}
\end{quadro}

\section{Comunidades}

A presença de comunidades como uma propriedade de redes complexas está tipicamente atrelada a ideia de haverem mais arestas internas a uma comunidade do que arestas que atravessem a fronteira entre duas comunidades.
Esse entendimento é solidificado na função de modularidade $Q$ de \citeonline{girvan2002community}, conforme descrito na equação \ref{eq:mod}.
\citeonline{shen2009detect} descreve uma adaptação, modularidade estendida $EQ$, para a aplicação em coberturas (em oposição a partições), conforme equação \ref{eq:mod_e}.
Essa aplicação em coberturas destaca a possibilidade de comunidades sobrepostas.
A aplicação de $EQ$ recursivamente, de forma a considerar o subgrafo com os membros da comunidade dividido entre as comunidades que a compõe descreve a propriedade de comunidades hierárquicas.

\begin{equation}
O_i = \left[\sum_{c \in C_n| L_{c} = i-1}V_c\right] - \left[\sum_{c \in C_n| L_{c} = i}V_c\right]
\end{equation}

A presença de sobreposição nas comunidades pode ser medida como um função $O_i$ que calcula a quantidade de vértices presentes em mais de uma comunidade de nível $i$.
O valor é dado pela diferença da quantidade de vértices em comunidades de nível $i$ e a quantidade de vértices em comunidades de nível $i+1$.
O efeito disso é que vértices que estejam em duas comunidades distintas de nível $i+1$ ou menores, serão contados duas vezes, mas os que pertencem a duas comunidades de nível $i$ ou menores serão compensadas.
Por fim, a soma de $O_i$ para todos os níveis será anotada como  $O$.

Observa-se na \autoref{tab:mod_base_params} os valores de $EQ$ para 10 execuções do modelo com os parâmetros básicos ($B_0$ a $B_9$).
Para as comunidades de primeiro nível, isso é, as que são compostas por sub comunidades, a função gera uma média de aproximadamente 0,8 o que indica uma estrutura de comunidade muito expressiva.
No segundo nível, o valor médio foi de aproximadamente 0,6 indicando comunidades menos expressivas mas ainda presentes.

\begin{table}[htbp]
    \centering
    \caption{Modularidade com os parâmetros básicos}
    \label{tab:mod_base_params}
    \begin{tblr}{l|r|r|r|r} \hline
        \SetCell{c} Execução & \SetCell{c} $EQ_1$ & \SetCell{c} $EQ_2$  & \SetCell{c} $O_1$ & \SetCell{c} $O_2$\\ \hline
        $B_0$ & 0.71695 & 0.79204 & 16 & 235 \\ \hline
        $B_1$ & 0.72441 & 0.79760 & 18 & 211 \\ \hline
        $B_2$ & 0.70039 & 0.79391 & 14 & 252 \\ \hline
        $B_3$ & 0.73633 & 0.79622 & 13 & 190 \\ \hline
        $B_4$ & 0.71758 & 0.78549 & 26 & 235 \\ \hline
        $B_5$ & 0.73179 & 0.79201 & 33 & 224 \\ \hline
        $B_6$ & 0.74779 & 0.79828 & 15 & 213 \\ \hline
        $B_7$ & 0.73912 & 0.80411 & 20 & 200 \\ \hline
        $B_8$ & 0.74237 & 0.80504 & 19 & 194 \\ \hline
        $B_9$ & 0.71339 & 0.79877 & 17 & 237 \\ \hline
        Média & 0.73701 & 0.79635 & 19 & 219 \\ \hline
    \end{tblr}
    \fonte{elaborado pelo autor}
\end{table}

A \autoref{tab:mod_base_params} indica também os valores de sobreposição entre comunidades de primeiro ($O_1$) e segundo nível ($O_2$).
Os valores indicam quem em média, dos mil vértices presentes em um grafo como o observado, dezessete estarão em duas comunidades de primeiro nível e em média duzentos e vinte e cinto vértices estarão em duas comunidades de segundo nível.
O que esses valores indicam é que até vinte e quatro por cento dos vértices, para essa parametrização, estariam distribuídos em mais de uma comunidade.

Naturalmente, a escolha de valores distintos para os parâmetros pode ser feita para reforçar as características desejáveis.
Limitando o valor de $MTE$ e considerando uma estrutura de comunidades gerada com  $K = (9, 2, 2, 2)$, obteve-se os dados distritos na \autoref{tab:mod_pro_params}.
Observa-se que a redução do número mínimo foi feita para evitar a descaracterização das comunidades folha, considerando que o valor de $K$ gera 72 dessas.

\begin{table}[htbp]
    \centering
    \caption{Modularidade com $K = (9, 2, 2, 2)$}
    \label{tab:mod_pro_params}
    \begin{tblr}{l|r|r|r|r|r} \hline
        \SetCell{c} Execução & \SetCell{c} $EQ_1$ & \SetCell{c} $EQ_2$ & \SetCell{c} $EQ_3$ & \SetCell{c} $EQ_4$ & \SetCell{c} $O$ \\ \hline
        $M_0$ & 0.96702 & 0.95833 & 0.93128 & 0.87673 & 0 \\ \hline
        $M_1$ & 0.96721 & 0.95979 & 0.93388 & 0.88221 & 0 \\ \hline
        $M_2$ & 0.96514 & 0.95933 & 0.93357 & 0.87722 & 0 \\ \hline
        $M_3$ & 0.96541 & 0.95711 & 0.92974 & 0.87888 & 0 \\ \hline
        $M_4$ & 0.96672 & 0.95868 & 0.93582 & 0.88090 & 0 \\ \hline
        Média & 0.96630 & 0.95865 & 0.93286 & 0.87919 & 0 \\ \hline
    \end{tblr}
    \fonte{elaborado pelo autor}
\end{table}

O incremento nos valores obtidos é atribuível à existência de mais comunidades para as quais distribuir os vértices, não apenas em profundidade mas em diâmetro também.
Descartando-se a possibilidade de ser a alteração do valor de $MTE$ ao realizar testes com este zerado mas mantendo o $K$ padrão.
Em compensação, essa parametrização faz com que a escolha de comunidades folha se torne muito mais provável, anulando a possibilidade de comunidades sobrepostas.
Esse comportamento ocorre por no processo de escolha das comunidades a ordenação das comunidades se dá por escolher quais comunidades tem os representantes mais próximos, e as comunidades folha por serem em maior quantidade são muito mais prováveis de serem escolhidas.

Outros testes descritos na \autoref{tab:mod_var_params}, onde com $MTE = 0$ realizou-se uma exploração dos valores de $EQ$ para diversos valores em $K$.
Note-se o caso $M_9$ como contra exemplo do impacto de $K$ na modularidade, bem como o exemplo $M_6$, que nas comunidades de primeiro teme $EQ = 0,5$.

\begin{table}[htbp]
    \centering
    \caption{Modularidade com variação de $K$}
    \label{tab:mod_var_params}
    \begin{tblr}{l|r|r} \hline
        \SetCell{c} Execução & $K$ & \SetCell{c} Média de $EQ$ \\ \hline
        % (0.59322 +0.94915 +0.79429)/3 = 0.778887
        $M_5$ & $(5, 5, 5)$ & 0.77889 \\ \hline 
        %(0.96943 +0.86339 +0.49881)/3 = 0.77721
        $M_6$ & $(2, 4, 8)$ & 0.77721 \\ \hline
        % (0.96947 +0.93266 +0.87089)/3 = 0.92434
        $M_7$ & $(8, 2, 4)$ & 0.92434 \\ \hline
        $M_8$ & $(100)$ & 0.95152 \\ \hline
        $M_9$ & $(3)$ & 0.49024 \\ \hline
    \end{tblr}
    \fonte{elaborado pelo autor}
\end{table}

Alternativamente, é possível modificar significativamente os valores obtidos para a modularidade com os parâmetros $E_\text{btw}^\text{max}$ e $E_\text{wth}^\text{max}$.
% (0.48115 + 0.4421)/2 = 0.461625
Na execução $M_{10}$ utilizou-se os demais parâmetros nos valores padrão, mas com $E_\text{btw}^\text{max}=45$ e $K = (2, 2)$, para um $EQ$ médio de 0,46.

\section{Homofilia e Homogeneidade}

Homogeneidade é a principal propriedade das comunidades definidas por semelhança de vértices, ela está intimamente ligada à propriedade de homofilia.
Assumindo a distância euclideana como função de semelhança, a forma natural de medir a homogeneidade das comunidades descritas por uma cobertura é a inercia e a razão da inercia entre diferentes níveis de comunidades.

\subsection{Inercia}

Inercia, definida como a soma dos quadrados das distancias de cada membro de um conjunto com a média do conjunto, sumariza quão semelhantes ou dissemelhantes os membros são.
A razão entre a inercia em diferentes níveis é facilmente extrapolada do conceito de razão de inercia introduzido por \citeonline{largeron2015generating}.
Assumindo que as comunidades de uma rede complexa sejam homogêneas e hierárquicas, é natural que as comunidades mais específicas (sub comunidades), tenham menos diversidade que as comunidade mais generalistas.
Tendo em vista que a inercia calculada com uma cobertura deve levar em conta que vértices serão contados múltiplas vezes se estiverem presentes em múltiplas comunidades, a inercia para um nível $l$ segue descrita na equação \ref{eq:inercia_l}.
Considerando que $C_l$ é o conjunto de comunidades de nível  $l$.

\begin{equation}\label{eq:inercia_l}
    I_l = \sum_{c \in C_l}\frac{\sum_{v \in c} |v-g_c|^2}{|c|}
\end{equation}

$I_l$ apresenta uma forma de comparação entre diferentes coberturas, de forma a identificar qual seria mais homogênea dado um mesmo conjunto do vértices.
A extensão do conceito de inercia para a razão entre a inercia aplicada a dois níveis distintos se dá para isolar o fator de escala.
Isso é, $\sfrac{I_l}{I_0}$ descreve não apenas quão homogêneo são os agrupamentos estabelecidos, mas o quão mais homogêneos eles são em relação à população em geral.
Note-se que $I_0$ nesse contexto se refere à comunidade $C_n$, isso é, a comunidade que contém todos os vértices.

\subsection{Distância esperada}

A homofilia, como propriedade de redes complexas onde os vértices tendem a estarem adjacentes à vértices semelhantes.
Em um sistema com atributos discretos, é facilmente estabelecido uma função $\Delta$  que descreve o quão mais semelhantes os vértices adjacentes são do que seria esperado em um modelo nulo.
O modelo nulo trivial neste caso é a consideração de vértices conectados com igual probabilidade, mantendo-se a mesma quantidade de vértices e arestas, conforme descrito por \citeonline{easley2010networks}.

A adaptação para um modelo de um espaço contínuo de múltiplas dimensões se dá calculando uma aproximação da distância média entre cada par de vértices, essa média, $H_e$, é o valor que seria esperado se o grafo não fosse homofílico.
O valor real é calculado como a média da distância dos pontos adjacentes, $H_r$.

\subsection{Resultados}

Na \autoref{tab:iner_base_params} estão discriminados os valores de inércia médios para as comunidades organizadas por nível.
A leitura que se pode fazer desses dados, em especial das médias, é: em média uma comunidade de primeiro nível tem 38\% da diversidade que se encontra na população geral; e em média uma comunidade de segundo nível (sub comunidade) tem 33\% da diversidade presente na população em geral.
Por fim, em média a distância entre dos vértices adjacentes é metade do que seria esperado sem a homofilia, indicando a presença dessa propriedade.

\begin{table}[htbp]
    \centering
    \caption{Homogeneidade e homofilia com os parâmetros básicos}
    \label{tab:iner_base_params}
    \begin{tblr}{l|r|r|r|r} \hline
        \SetCell{c} Execução & \SetCell{c} $\sfrac{I_1}{I_0}$ & \SetCell{c} $\sfrac{I_2}{I_0}$ & \SetCell{c} $\sfrac{I_1+I_2}{2I_0}$ & \SetCell{c} $\sfrac{H_r}{H_e}$
        \\ \hline
$B_0$ & 0.38193 & 0.34155 & 0.36174 & 0.45942 \\ \hline
$B_1$ & 0.34067 & 0.30168 & 0.32118 & 0.45920 \\ \hline
$B_2$ & 0.34262 & 0.29697 & 0.31980 & 0.49265 \\ \hline
$B_3$ & 0.35961 & 0.29986 & 0.32974 & 0.48305 \\ \hline
$B_4$ & 0.41670 & 0.37592 & 0.39631 & 0.52725 \\ \hline
$B_5$ & 0.35321 & 0.28246 & 0.31784 & 0.46047 \\ \hline
$B_6$ & 0.37795 & 0.29310 & 0.33553 & 0.46127 \\ \hline
$B_7$ & 0.35209 & 0.29903 & 0.32556 & 0.45348 \\ \hline
$B_8$ & 0.36251 & 0.32820 & 0.34536 & 0.46364 \\ \hline
$B_9$ & 0.36874 & 0.32936 & 0.34905 & 0.50499 \\ \hline
Média & 0.36560 & 0.31481 & 0.34021 & 0.47654 \\ \hline
    \end{tblr}
    \fonte{elaborado pelo autor}
\end{table}

Naturalmente, existe uma influencia significativa que a quantidade de comunidades tem neste valor.
Observe-se na \autoref{tab:iner_pro_params}, com mais comunidades para distribuir os vértices, as comunidades ficam significativamente mais homogêneas.
Como efeito disso, a distância entre vértices conectados é bastante reduzida.

\begin{table}[htbp]
    \centering
    \caption{Homofilia e Homogeneidade com $K = (9, 2, 2, 2)$}
    \label{tab:iner_pro_params}
    \begin{tblr}{l|r|r|r|r|r} \hline
        \SetCell{c} Execução & \SetCell{c} $\sfrac{I_1}{I_0}$ & \SetCell{c} $\sfrac{I_2}{I_0}$ & \SetCell{c} $\sfrac{I_3}{I_0}$ & \SetCell{c} $\sfrac{I_4}{I_0}$ & \SetCell{c} $\sfrac{H_r}{H_e}$ \\ \hline
        $M_0$ & 0.18740 & 0.11020 & 0.05650 & 0.03240 & 0.15975 \\ \hline
        $M_1$ & 0.18766 & 0.11563 & 0.06017 & 0.03277 & 0.16491 \\ \hline
        $M_2$ & 0.18446 & 0.11812 & 0.05993 & 0.03526 & 0.17281 \\ \hline
        $M_3$ & 0.18566 & 0.11307 & 0.06003 & 0.03169 & 0.16211 \\ \hline
        $M_4$ & 0.18111 & 0.11163 & 0.05563 & 0.03016 & 0.16158 \\ \hline
        Média & 0.18526 & 0.11373 & 0.05845 & 0.03246 & 0.16423 \\ \hline
    \end{tblr}
\fonte{elaborado pelo autor}
\end{table}

% (0.18740 + 0.18766 + 0.18446 + 0.18566 + 0.18111)/5 = 0.18526
% (0.11020 + 0.11563 + 0.11812 + 0.11307 + 0.11163)/5 = 0.11373
% (0.05650 + 0.06017 + 0.05993 + 0.06003 + 0.05563)/5 = 0.05845
% (0.03240 + 0.03277 + 0.03526 + 0.03169 + 0.03016)/5 = 0.03246
% (0.15975 + 0.16491 + 0.17281 + 0.16211 + 0.16158)/5 = 0.16423

Também de forma natural, existe uma relação entre o parâmetro $\theta$ e a inercia.
É apontado na tabela \autoref{tab:iner_var_params} a variação da inercia conforme se altera $\theta$.
Duas características dos dados descritos são relevantes, primeiramente com $\theta = 1$ se a ortogonalidade das comunidades não estivesse tendo efeito, seria de se esperar comunidades onde a diversidade interna se aproximaria da diversidade da população geral, mas não é isso o que se observa.
Isso é, em um modelo nulo, onde não existisse correlação entre os atributos do vértice e a comunidade em que ele participa, seria esperado uma razão de inercia que tendesse as um.

\begin{table}[htbp]
    \centering
    \caption{Homofilia e Homogeneidade com $\theta$ variável}
    \label{tab:iner_var_params}
    \begin{tblr}{l|r|r|r|r} \hline
        \SetCell{c} Execução & \SetCell{c} $\theta$ & \SetCell{c} $\sfrac{I_1}{I_0}$ & \SetCell{c} $\sfrac{I_2}{I_0}$ &
        \SetCell{c} Média
        \\ \hline
        $I_0$ & \multirow{3}{*}{1.00} & 0.78255 & 0.73782 & \multirow{3}{*}{0.71637} \\ \hline
        $I_1$ &                       & 0.70558 & 0.67610 &                          \\ \hline
        $I_2$ &                       & 0.70773 & 0.68841 &                          \\ \hline
        $I_3$ & \multirow{3}{*}{0.66} & 0.40448 & 0.34173 & \multirow{3}{*}{0.36539} \\ \hline
        $I_4$ &                       & 0.40280 & 0.33749 &                          \\ \hline
        $I_5$ &                       & 0.37041 & 0.33541 &                          \\ \hline
        $I_6$ & \multirow{3}{*}{0.33} & 0.34582 & 0.29460 & \multirow{3}{*}{0.34005} \\ \hline
        $I_7$ &                       & 0.36632 & 0.30172 &                          \\ \hline
        $I_8$ &                       & 0.40126 & 0.33059 &                          \\ \hline
    \end{tblr}
\fonte{elaborado pelo autor}
\end{table}


O segundo ponto a ser destacado é que $\theta$ não é um controle direto sobre a inercia, nos casos com  $\theta=0.66$, $\theta=0.33$ e $\theta=0.1$ (\autoref{tab:iner_base_params}), a razão se manteve em um mesmo patamar, entre 34\% e 40\%.
Existem outros fatores que afetam diretamente a homogeneidade das comunidades, e que são mais dificilmente controlados por parâmetros.

\section{Visualização}

Uma das características da propriedade de existência de comunidades é que a definição de comunidade é dependente de contexto.
Ainda assim, parece haver um entendimento natural de comunidades e de suas propriedades a partir da visualização das mesmas.
Segue a baixo a renderização dos grafos $B_0$ e $M_0$, gerados respectivamente com os parâmetros padrão e com os parâmetros otimizando para a modularidade (mais valores em $K$ e valores maiores).

A renderização desses grafos conta com uma codificação de forma que vértices que estejam na mesma comunidade de primeiro nível compartilhem o formato, e vértices que estejam na mesma comunidade de segundo nível compartilhem as cores, como é observável na \autoref{fig:m0}.
Vértices que estejam em mais de uma comunidade de segundo nível terão a cor branca, e vértices que estejam em mais de uma comunidade de primeiro nível terão o formato de círculo.
Dessa forma, é possível identificar as áreas de sobreposição, bem como os primeiros níveis da hierarquia de comunidades.

\begin{figure}[htpb]
    \centering
    \caption{$M_0$ renderizado aproximando os vértices adjacentes}\label{fig:m0}
    \includegraphics[width=\textwidth, height=0.6\textheight]{figures/m0.png}
    \fonte{elaborado pelo autor.}
\end{figure}

Na \autoref{fig:m0} está a renderização do grafo $M_0$ com o posicionamento dos vértices de forma a minimizar a distância deles na imagem.
Além das divisões entre comunidades que estão categorizadas com os formatos e com as cores, é visível uma divisão interna de pequenos grupos densamente conexos.
Essa visualização e entendimento intuitivo da divisão é decorrente de uma presença muito acentuada da propriedade de comunidades.

Essa divisão mais presenta, conforme já foi distrito anteriormente, é responsável também pela falta de vértices em regiões sobrepostas, de certa forma sacrificando ganhos nessa propriedade (sobreposição) para ganhar maior definição em outra (comunidades hierárquicas).
A \autoref{fig:m0_pos} e a \autoref{fig:m0_pos_wires} demonstram isso.
Ambas apresentam os vértices posicionados com base nos seus atributos, de forma que o centro (mais próximo a origem) e mais densamente povoado, no entanto a \autoref{fig:m0_pos_wires} não renderiza os vértices, apenas as arestas.
Com isso, é abundantemente claro o quão fortemente delineadas são as comunidades.
Demonstra-se que cada comunidade possui um domínio absoluto sobre alguma região do espaço onde os vértices se encontram.

\begin{figure}[htpb]
    \centering
    \caption{$M_0$ renderizado com os vértices posicionado pelos seus atributos}\label{fig:m0_pos}
    \includegraphics[width=0.8\textwidth, height=0.32\textheight]{figures/m0_pos.png}
    \fonte{elaborado pelo autor.}
\end{figure}

\begin{figure}[htpb]
    \centering
    \caption{$M_0$ renderizado com os vértices posicionado pelos seus atributos (apenas arestas)}\label{fig:m0_pos_wires}
    \includegraphics[width=\textwidth, height=0.52\textheight]{figures/m0_pos_wires.png}
    \fonte{elaborado pelo autor.}
\end{figure}

A \autoref{fig:b0_pos} e a \autoref{fig:b0_pos_wires} fazem a mesma distinção, mas bom o grafo $B_0$.
$B_0$ apresenta um conjunto de comunidades onde ocorre a sobreposição, não apenas no sentido de comunidades sobrepostas, mas em um sentido mais simples, de que há vértices que não pertencem a comunidade (e que não tem ligações com ela) ocupando o mesmo espaço.
Novamente, esse é um efeito de uma menor proporção de comunidades.

A visualização das arestas mostra uma divisão visual das comunidades, com uma comunidade no centro e cinco comunidades rodeando essa.
Dessas seis comunidades, é possível identificar que existem grupos distintos dentro delas, isso é, as sub comunidades são visíveis, mas pela natureza delas (e presença de vértices que certem as duas), são divisões muito menos precisas.

\begin{figure}[htpb]
    \centering
    \caption{$B_0$ renderizado com os vértices posicionado pelos seus atributos}\label{fig:b0_pos}
    \includegraphics[width=\textwidth, height=0.52\textheight]{figures/b0_pos.png}
    \fonte{elaborado pelo autor.}
\end{figure}

\begin{figure}[htpb]
    \centering
    \caption{$B_0$ renderizado com os vértices posicionado pelos seus atributos (apenas arestas)}\label{fig:b0_pos_wires}
    \includegraphics[width=0.8\textwidth, height=0.32\textheight]{figures/b0_pos_wires.png}
    \fonte{elaborado pelo autor.}
\end{figure}

% \begin{figure}[htpb]
%     \centering
%     \caption{Exemplo de grafo}\label{fig:b0}
%     \includegraphics[width=\textwidth, height=0.42\textheight]{figures/b0.png}
%     \fonte{elaborado pelo autor.}
% \end{figure}

% \begin{figure}[htpb]
%     \centering
%     \caption{Exemplo de grafo}\label{fig:b0_wires}
%     \includegraphics[width=1\textwidth, height=0.42\textheight]{figures/b0_wires.png}
%     \fonte{elaborado pelo autor.}
% \end{figure}

\end{document}
