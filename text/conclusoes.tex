\documentclass[notes.tex]{subfiles}
\begin{document}

\chapter{Conclusões}

Este trabalho demonstra um modelo algorítmico para a geração de redes complexas com comunidades hierárquicas e comunidades sobrepostas.
Junto do algoritmo foram identificados valores básicos para a parametrização que geram grafos com todas as propriedades desejadas.
A fundamentação teórica apresenta uma exploração em língua portuguesa academicamente relevante, descrevendo redes complexas, comunidades e as suas propriedades.
Os resultados experimentais demonstram o sucesso nos objetivos gerais e específicos estipulados para o modelo.

Quando ao objetivo específico da inclusão de comunidades na rede gerada, bem como a construção de uma \emph{ground truth} para quais são os membros das comunidades é demonstrado como bem sucedido na exploração dos resultados da função de modularidade estendida $EQ$.
A presença das comunidades, bem como o pertencimento dos vértices a elas é confirmado pela função de qualidade de cobertura.
Notadamente, essa função e a sua aplicação recursiva destacam a presença de comunidades hierarquicamente organizadas, de forma que se aninham em comunidades auto semelhantes, e sobrepostas, de forma a compartilharem vértices.
Demonstradamente, o compartilhamento de vértices é sensível à parametrização do modelo, se mostrando mais presente em grafos gerados a partir de parâmetros que geram uma estrutura com menos comunidades.

O objetivos relativos à comunidades homogêneas foram bem sucedidos conforme demonstrado experimentalmente, sendo parametrizável o quão pronunciada é essa propriedade.
Com a escolha de parâmetros que reforcem comunidades de tamanho reduzido (maiores valores para $K$) os grafos gerados consistentemente apresentam comunidades com menos de quatro porcento da diversidade global.
Os parâmetros básicos, consideravelmente conservadores no quanto eles reforçam características específicas, geram grafos onde as comunidades apresentam consistentemente menos de trinta e quatro por cento da diversidade que o grafo.

Por fim, acredita-se que esse trabalho poderá se tornar uma referência relevante, dada a falta de literatura em língua portuguesa para a área de redes complexas e comunidades.

\section{Extensões}

Redes complexas como área de estudo é um campo promissor e mais estudos devem ser realizados.
O processo de geração de redes complexas com comunidades é uma área de relevância acadêmica, e o modelo apresentado neste trabalho tem um conjunto de extensões desejáveis, são elas:

A otimização da cobertura gerada para algorítimos de percolação de cliques, por meio da redução de arestas entre vértices que estejam entre duas comunidades.
Isso poderia ser implementado no processo de geração final de arestas, removendo estes vértices e não os considerando na avaliação de que a comunidade é um grafo completo.
Idealmente, isso deveria ser parametrizado com um valor entre zero e um, identificando a proporção de arestas entre vértices na mesma área de superposição.

A elaboração de um método mais deliberado nas escolha de comunidades, que permita um controle mais fino da proporção de vértices em áreas de superposição.
Uma implementação intuitiva seria a parametrização de uma proporção, quantos vértices estariam em uma comunidade folha apenas, quantos estariam em duas, quantos estariam em três (O modelo não é incompatível com a escolha de três ou mais comunidades) etc.
Cada vértice teria primeiramente uma quantidade de comunidades folha decidido com essas proporções, e a escolha poderia ser feita ou repetindo a chamada para a função, ou escolhendo as comunidades mais próximas.

O desenvolvimento de novos modelos de eleição de representantes, considerando que naturalmente a escolha de representantes com características distintas pode otimizar o processo.
As escolhas de metodologia naturais seriam a minimação da função $d$ para os vértices já dentro da comunidade, tanto considerando a média dos vértices para com os representantes eleitos, quanto para o seu mínimo.

O desenvolvimento de novos modelos para a função $d$, especialmente em vista da ortogonalidade.
Uma abordagem que se desejaria explorar é a limitação da quantidade de níveis hierárquicos para a quantidade de dimensões nos vértices, para que sejam usados eixos distintos como eixo de ortogonalidade por nível.
Isso é, as comunidades de primeiro nível teriam predileção por discriminar seus membros com base em um eixo, as de segundo nível teriam por outro eixo (ortogonal ao primeiro) e assim sucessivamente.
De fato, a implementação de uma estrutura assim poderia suplantar a necessidade de um parâmetro de controle  $\theta$, pois a variação em  $\A$ seria capaz de modificar esse comportamento controlando inclusive para diferentes níveis.

Um estudo mais aprofundado das proporcionalidades identificadas em sistemas do mundo real e a sua presença nos grafos gerados pelo modelo.
Um exemplo disso é a distribuição de graus em função logarítmica, que apesar de ter sido observado em testes, não pode ser avalada mais rigorosamente.

Uma avaliação comparativa da cobertura gerada pelo modelo com as coberturas identificadas por algoritmos de identificação de comunidades canonicamente aceitos na literatura.

\end{document}
