%% abtex2-modelo-artigo.tex, v<VERSION> laurocesar
%% Copyright 2012-<COPYRIGHT_YEAR> by abnTeX2 group at http://www.abntex.net.br/
%%
%% This work may be distributed and/or modified under the
%% conditions of the LaTeX Project Public License, either version 1.3
%% of this license or (at your option) any later version.
%% The latest version of this license is in
%%   http://www.latex-project.org/lppl.txt
%% and version 1.3 or later is part of all distributions of LaTeX
%% version 2005/12/01 or later.
%%
%% This work has the LPPL maintenance status `maintained'.
%%
%% The Current Maintainer of this work is the abnTeX2 team, led
%% by Lauro César Araujo. Further information are available on
%% http://www.abntex.net.br/
%%
%% This work consists of the files abntex2-modelo-artigo.tex and
%% abntex2-modelo-references.bib
%%

% ------------------------------------------------------------------------
% ------------------------------------------------------------------------
% abnTeX2: Modelo de Artigo Acadêmico em conformidade com
% ABNT NBR 6022:2018: Informação e documentação - Artigo em publicação
% periódica científica - Apresentação
% ------------------------------------------------------------------------
% ------------------------------------------------------------------------

\documentclass[
    % -- opções da classe memoir --
    % article,            % indica que é um artigo acadêmico
    11pt,               % tamanho da fonte
    oneside,            % para impressão apenas no recto. Oposto a twoside
    a4paper,            % tamanho do papel.
    % -- opções da classe abntex2 --
    chapter=TITLE,     % títulos de capítulos convertidos em letras maiúsculas
    section=TITLE,     % títulos de seções convertidos em letras maiúsculas
    % subsection=TITLE,  % títulos de subseções convertidos em letras maiúsculas
    %subsubsection=TITLE % títulos de subsubseções convertidos em letras maiúsculas
    % -- opções do pacote babel --
    english,            % idioma adicional para hifenização
    brazil,             % o último idioma é o principal do documento
    sumario=tradicional
    ]{abntex2}


% ---
% PACOTES
% ---

% ---
% Pacotes fundamentais
% ---
\usepackage{lmodern}            % Usa a fonte Latin Modern
\usepackage[T1]{fontenc}        % Selecao de codigos de fonte.
\usepackage[utf8]{inputenc}     % Codificacao do documento (conversão automática dos acentos)
\usepackage{indentfirst}        % Indenta o primeiro parágrafo de cada seção.
\usepackage{nomencl}            % Lista de simbolos
\usepackage{color}              % Controle das cores
\usepackage{graphicx}           % Inclusão de gráficos
\usepackage{microtype}          % para melhorias de justificação
\usepackage{setspace}
\usepackage{tocloft}
\usepackage{enumitem}
% ---

% ---
% Pacotes adicionais, usados apenas no âmbito do Modelo Canônico do abnteX2
% ---
\usepackage{lipsum}             % para geração de dummy text
% ---

% ---
% Pacotes de citações
% ---
\usepackage[brazilian,hyperpageref]{backref}     % Paginas com as citações na bibl
\usepackage[alf]{abntex2cite}   % Citações padrão ABNT
% ---

% ---
% Configurações do pacote backref
% Usado sem a opção hyperpageref de backref
\renewcommand{\backrefpagesname}{Citado na(s) página(s):~}
% Texto padrão antes do número das páginas
\renewcommand{\backref}{}
% Define os textos da citação
\renewcommand*{\backrefalt}[4]{
    \ifcase #1 %
        Nenhuma citação no texto.%
    \or
        Citado na página #2.%
    \else
        Citado #1 vezes nas páginas #2.%
    \fi}%
% ---

% ---
% Pacotes de utilidades, para legibilidade
% ---
\usepackage{subfiles}
% ---

% ---
% Configurações do abntex
% ---
\renewcommand{\ABNTEXchapterfont}{\rmfamily}
\renewcommand{\ABNTEXchapterfontsize}{\normalsize \bfseries} % solução deselegantes, mas funcional pra deixar o titulo das sessões em negrito
\renewcommand{\ABNTEXsectionfontsize}{\normalsize}


\addto\captionsbrazil{
\renewcommand{\listfigurename}{Lista de Figuras}
}
% ---


% --- Informações de dados para CAPA e FOLHA DE ROSTO ---
\titulo{Geração de redes complexas com comunidades sobrepostas e comunidades hierárquicas}
\tituloestrangeiro{Complex network generation with overlapping communities and hierarchical communities}
\autor{Gustavo Henrique Spiess}
\local{Blumenau}
\data{2022}
\orientador{Aurelio Faustino Hoppe}
\instituicao{%
    Universidade regional de Blumenau 
    \protect \\
    Centro de ciências exatas e naturais 
    \protect \\
    Curso de ciência da computação -- Bacharelado}
\tipotrabalho{Monografia}
\preambulo{
Geração de redes complexas
}
% ---

% ---
% Configurações de aparência do PDF final

% alterando o aspecto da cor azul
\definecolor{blue}{RGB}{41,5,195}

% informações do PDF
\makeatletter
\hypersetup{
        %pagebackref=true,
        pdftitle={\@title},
        pdfauthor={\@author},
        pdfsubject={\imprimirpreambulo},
        pdfcreator={LaTeX with abnTeX2},
        pdfkeywords={Redes complexas}{Geração de redes complexas}{Comunidades}{Comunidades sobrepostas}{Comunidades hierárquicas},
        colorlinks=true,            % false: boxed links; true: colored links
        linkcolor=blue,             % color of internal links
        citecolor=blue,             % color of links to bibliography
        filecolor=magenta,              % color of file links
        urlcolor=blue,
        bookmarksdepth=4
}
\makeatother

% ---
% Redefinição de comandos para adequar ao formato requerido
% ---
\makeatletter
\renewcommand{\imprimircapa}{

    \DoubleSpacing

    \begin{centering}

        {\bfseries \MakeUppercase{\imprimirinstituicao}\par}
    
        \vspace*{\fill} \vspace*{\fill} \vspace*{\fill} \vspace*{\fill}
        \vspace*{\fill} \vspace*{\fill} \vspace*{\fill}
    
        {\bfseries\LARGE\MakeUppercase{\imprimirtitulo}}\\
    
        \vspace*{\fill}

        \begin{flushright}
            \bfseries \MakeUppercase{\imprimirautor}
        \end{flushright}   

        \vspace*{\fill} \vspace*{\fill} \vspace*{\fill} \vspace*{\fill}
        \vspace*{\fill} \vspace*{\fill}

        \SingleSpace

        {\bfseries \MakeUppercase{\imprimirlocal} \\ \MakeUppercase{\imprimirdata} }
        \vspace*{\fill}

    \end{centering}
    \pagebreak
}
\makeatother

% ---

\makeatletter
\renewcommand{\imprimirfolhaderosto}{

    \SingleSpace
    \begin{centering}

        {\bfseries \MakeUppercase{\imprimirautor}\par}
    
        \vspace*{\fill} \vspace*{\fill}
    
        {\bfseries\LARGE\MakeUppercase{\imprimirtitulo}}\\
    
        \vspace*{\fill}

        \begin{flushright}
            \parbox{0.5\textwidth}{%
                Trabalho de Conclusão de Curso apresentado ao curso de graduação em Ciências da Computação no Centro de de Ciências Exatas e Naturais da Universidade Regional de Blumenau como requisito parcial para a obtenção de grau de Bacharel em Ciências da Computação.
            }
        \end{flushright}   

        \begin{flushright}
            \parbox{\textwidth*3/5}{%
                Professor \imprimirorientador, Titulação - Orientador
            }
        \end{flushright}   

        \vspace*{\fill} \vspace*{\fill} \vspace*{\fill}
        \vspace*{\fill} \vspace*{\fill} \vspace*{\fill}

        {\bfseries \MakeUppercase{\imprimirlocal} \\ \MakeUppercase{\imprimirdata} }

    \end{centering}
    \pagebreak
}
\makeatother

% ---

\makeatletter
\newcommand{\imprimirfolhadeassinaturas}{
    \SingleSpace
    \begin{centering}
        \vspace*{\fill} 
        {\bfseries\LARGE\MakeUppercase{Folha de assinaturas}} 
        %\includepdf{folhadeaprovacao_final.pdf}
        \vspace*{\fill}
    \end{centering}
    \pagebreak
}
\makeatother

% ---

% Novo list of (listings) para QUADROS

\newcommand{\quadroname}{Quadro}
\newcommand{\listofquadrosname}{Lista de quadros}

\newfloat[chapter]{quadro}{loq}{\quadroname}
\newlistof{listofquadros}{loq}{\listofquadrosname}
\newlistentry{quadro}{loq}{0}

% configurações para atender às regras da ABNT
\setfloatadjustment{quadro}{\centering}
\counterwithout{quadro}{chapter}
\renewcommand{\cftquadroname}{\quadroname\space} 
\renewcommand*{\cftquadroaftersnum}{\hfill--\hfill}

% Configuração de posicionamento padrão:
\setfloatlocations{quadro}{hbtp}

% ---

\renewenvironment{siglas}{%
  \pretextualchapter{\listadesiglasname}
  \begin{itemize}[label={}]
}{%
  \end{itemize}
  \cleardoublepage
}

% ---


% ---
% compila o indice
% ---
\makeindex
% ---

% ---
% Altera as margens padrões
% ---
\setlrmarginsandblock{3cm}{2cm}{*}
\setulmarginsandblock{3cm}{2cm}{*}
\checkandfixthelayout
% ---

% ---
% Espaçamentos entre linhas e parágrafos
% ---

% O tamanho do parágrafo é dado por:
\setlength{\parindent}{1.3cm}

% Controle do espaçamento entre um parágrafo e outro:
\setlength{\parskip}{0.2cm}  % tente também \onelineskip
\setlength{\afterchapskip}{1.4cm}


% ----
% Início do documento
% ----
\begin{document}

\selectlanguage{brazil}
\frenchspacing

% ----------------------------------------------------------
% ELEMENTOS PRÉ-TEXTUAIS
% ----------------------------------------------------------

% página de titulo principal (obrigatório)
\imprimircapa

\imprimirfolhaderosto

\imprimirfolhadeassinaturas

\begin{dedicatoria}
    \vspace*{\fill}
    \hfill
    \parbox{0.5\textwidth}{%
        \lipsum[2]
    }
    \vspace*{\fill}
\end{dedicatoria}
\pagebreak

\DoubleSpacing
\begin{agradecimentos}
    \bigskip \bigskip
    \lipsum[1]
\end{agradecimentos}
\pagebreak

\SingleSpace
\begin{epigrafe}
    \vspace*{\fill}

    \hfill
    \parbox{0.5\textwidth}{%
    Frase inspiradora com significado maneiro e/ou importante
    }

    \begin{flushright}
    Autor da frase
    \end{flushright}
    \vspace*{\fill}
\end{epigrafe}

% resumo em português
\DoubleSpacing
\begin{resumoumacoluna}
\bigskip

\lipsum[1]

 \vspace{\onelineskip}

 \noindent
 \textbf{Palavras-chave}: Redes complexas. Geração de redes complexas. Comunidades. Comunidades sobrepostas. Comunidades hierárquicas.
\end{resumoumacoluna}


% resumo em inglês
\renewcommand{\resumoname}{Abstract}
\begin{resumoumacoluna}
 \begin{otherlanguage*}{english}
\bigskip

   \lipsum[2]

   \vspace{\onelineskip}

   \noindent
   \textbf{Keywords}: Complex networks. Complex networks generation. Communities. Overlapping communities. Hierarchical communities
 \end{otherlanguage*}
\end{resumoumacoluna}

\pdfbookmark[0]{\listfigurename}{lof}
\listoffigures*
\pagebreak

\listofquadros*
\pagebreak

\listoftables*
\pagebreak

\begin{siglas}
\item[IE] -- Internet Explorer
\item[QE] -- Qternet Explorer
\end{siglas}
\pagebreak

\tableofcontents* 


% \begin{center}\smaller % \textbf{Data de submissão e aprovação}: 14 de maio de 2022.
%
% \textbf{Identificação e disponibilidade}: \href{mailto:gustavospiess@gmail.com}{gustavospiess@gmail.com}.
% \end{center}

% ----------------------------------------------------------
% ELEMENTOS TEXTUAIS
% ----------------------------------------------------------
\textual

% ----------------------------------------------------------
% Introdução
% ----------------------------------------------------------
\chapter{Introdução}

\subfile{introducao}

\begin{figure}[h]
    \caption{\label{fig:exmp}Exemplo}
    Lorem
\end{figure}
\medskip

\begin{quadro}[htb]
\caption{\label{qua:exmp}Exemplo}
    Lorem
\end{quadro}

\begin{table}[htb]
    \centering
    \caption{\label{tab:exmp}Exemplo}
    Lorem
\end{table}

% ---
% Finaliza a parte no bookmark do PDF, para que se inicie o bookmark na raiz
% ---
\bookmarksetup{startatroot}%
% ---

% ---
% Conclusão
% ---
\chapter{Considerações finais}

\lipsum[5]

\begin{citacao}
\lipsum[6]
\end{citacao}

\lipsum[7]

% ---------------------------------------------------------- % ELEMENTOS PÓS-TEXTUAIS
% ----------------------------------------------------------
\postextual

% ----------------------------------------------------------
% Referências bibliográficas
% ----------------------------------------------------------
\bibliography{cngohc_ref}

% ----------------------------------------------------------
% Glossário
% ----------------------------------------------------------
%
% Há diversas soluções prontas para glossário em LaTeX.
% Consulte o manual do abnTeX2 para obter sugestões.
%
%\glossary

% ----------------------------------------------------------
% Apêndices
% ----------------------------------------------------------

% ---
% Inicia os apêndices
% ---
% \begin{apendicesenv}

% ----------------------------------------------------------
% \chapter{Nullam elementum urna vel imperdiet sodales elit ipsum pharetra ligula ac pretium ante justo a nulla curabitur tristique arcu eu metus}
% ----------------------------------------------------------
% \lipsum[55-56]

% \end{apendicesenv}
% ---

% ----------------------------------------------------------
% Anexos
% ----------------------------------------------------------
% \cftinserthook{toc}{AAA}
% ---
% Inicia os anexos
% ---
%\anexos
% \begin{anexosenv}

% ---
% \chapter{Cras non urna sed feugiat cum sociis natoque penatibus et magnis dis parturient montes nascetur ridiculus mus}
% ---

% \lipsum[31]

% \end{anexosenv}

% ----------------------------------------------------------
% Agradecimentos
% ----------------------------------------------------------

\chapter*{Agradecimentos}
\lipsum[9]

\end{document}
