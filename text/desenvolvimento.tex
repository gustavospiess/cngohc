\documentclass[notes.tex]{subfiles}

\begin{document}

A proposta de desenvolvimento desse trabalho é a extensão do modelo proposto por \citeonline{largeron2015generating} para a construção de comunidades hierárquicas e sobrepostas.
A modelagem proposta se baseia na construção de uma cobertura (em oposição á construção de uma partição) recursiva, de forma semelhante ao que é produzido pelo algoritmo proposto por \citeonline{shen2009detect}.

\section{O Modelagem}

O modelagem proposto é:

\begin{quadro}[htb]
\caption{\label{qua:modelagem}Modelagem de grafo com comunidades hierárquicas e sobrepostas}

    \begin{empheq}[box=\fbox]{align*}
        \G &= (\V, \E, C_n) \\
        \V &\subset  \mathbb{Q}^{p} \\
        \E &\subset \set{v_1, v_2 | v_1 \in \V, v_2 \in \V} \\
         C_0 &\subset \Set{v | v \in \V} \\
         C_n &= \Set{C_{n-1_0}, C_{n-1_1}, \ldots, C_{n-1_m}}
    \end{empheq}

    \fonte{elaborado pelo autor}
\end{quadro}

Os vértices em $\V$ são uma nuvem de pontos em um espaço de $p$ dimensões, cada um sendo um vetor com $p$ valores racionais (representados com ponto flutuante).
As arestas em $\E$ são simples pares não ordenados de vértices.
A cobertura $C_n$ é uma estrutura recursiva de conjuntos com $n$ níveis, onde o conjunto $C_n$ é composto por $m$ conjuntos $C_{n-1}$.
No último níveo o conjunto $C_0$ é composto por vértices do grafo, esses conjuntos são as comunidades \emph{folha}, significando que elas não são compostas por outras comunidades.

Essa estrutura recursiva é a representação das comunidades hierárquicas, onde os vértices que pertencem a uma comunidade $C$, denotado $V_C$ são os membros do conjunto união dos membros de  $C$, isso é, $\bigcup_{S \in c} V_S$.
Como característica dessa modelagem, os vértices da comunidade representada por $C_n$ são a totalidade dos vértices do grafo, portanto $V_{C_n} = \V$
Isso implica que para qualquer vértice em $\V$, ele deve de estar presente pelo menos uma comunidade folha.
Ao estar presente em uma comunidade folha, o vértice é considerando também parte de todas as comunidades compostas por esta comunidade folha.

A cardinalidade de cada um dos conjuntos que formam a cobertura é variável de acordo com o nível, isso é, dado um nível $x$ todas as coberturas  $C_{n-x}$ possuem a mesma quantidade de elementos, mas conjuntos de níveis distintos podem possuir quantidades de elementos distintas.
É considerada também a existência de um vetor, denotado $K$, em um espaço de $n-1$ dimensões, que denota a cardinalidade das coberturas compostas por outras coberturas.
Todas as comunidades do grafo $\G$ contém pelo menos um membro.

Uma característica de notação é a função $flat$ que mapeia uma comunidade para um conjunto de quais as comunidades que a compõe.
Para efeito de notação $flat(C_n)$ é um conjunto com todas as comunidades do grafo, incluindo a comunidade global.
E o mapa $L$, que relaciona cada comunidade com a quantidade de ancestrais que a comunidade possui, ou seja, com a quantidade de comunidades que ela compõe.
As notações utilizadas serão  $L_C$, leia-se nível de $C$, ou $L$, leia-se níveis.
A raiz da cobertura é o nível zero ($L_{C_n} = 0$) e a o nível de um nó folha é igual a $n$ ($L_{C_0}=n$)

As características que devem ser verdadeiras se um grafo for representado nesta modelagem encontram-se descritas na \autoref{tab:prop_modelagem}.

\begin{table}[htb]
\caption{Características da modelagem}
\label{tab:prop_modelagem}

    \begin{tblr}{X|l} \hline
        \SetCell{c} Característica & \SetCell{c} Formalismo
        \\ \hline
        Para toda a comunidade $C$, se ela não for folha, a função $flat$ dela é a união de $C$ com a função flat de seus componentes.&
        $\displaystyle \forall C (L_C < n \implies flat(C) = C\cup\bigcup_{S \in C}flat(S))$
        \\\hline
        Para toda a comunidade $C$, se ela for folha, a função $flat$ dela é um conjunto consigo.&
        $\displaystyle \forall C (L_C = n \implies flat(C) = \set{C})$
        \\\hline
        A comunidade raiz engloba todos os vértices do grafo.&
        $\displaystyle V_{C_n} =  \V$
        \\ \hline
        Para todas as comunidades $C$, se $C$ não for folha, os vértices englobados em C são a união dos vértices englobados em seus componentes&
        $\displaystyle \forall C (L_c > 0 \implies V_C =  \bigcup_{S \in C} V_S)$
        \\ \hline
        Para todas as comunidades $C$, se $C$ for folha, os vértices englobados em C seus componentes&
        $\displaystyle \forall C (L_C = 0 \implies V_{C_0} = C_0)$
        \\ \hline
        Para todos os vetores do grafo, existe uma comunidade folha a qual ela pertence &
        $\displaystyle \forall v (\exists C(v \in C \land L_C = n))$
        \\ \hline
        Pra todo vértice $v$, pra todo $l$, existe uma comunidade $C$ que contenha o vértice e seja do nível $l$&
        $\displaystyle \forall v \forall l (\exists C (v \in V_{C_x} \land L_C = l))$
        \\ \hline
        Todos as comunidades não folha tem a mesma quantidade de componentes se forem do mesmo nível, a cardinalidade de uma comunidade não folha é expressa num vetor $K$ &
        $\displaystyle \exists K (K \in \mathbb{I}^{n-1} \land \forall C (L_C < n \implies K_{L_C} = |C|))$
        \\ \hline
        Toda a comunidade tem pelo menos uma componente e engloba pelo menos um vértice&
        $\displaystyle \forall C (|C| \ge 1 \land |V_C| \ge 1)$
        \\ \hline
    \end{tblr}

    \fonte{elaborado pelo autor}
\end{table}

\section{Propriedades desejáveis}

\citeonline{largeron2015generating} implementa um modelo algorítmico de geração de redes complexas que mantém uma série de propriedades desejáveis.
Como a implementação proposta se baseia no modelo de \citeonline{largeron2015generating}, é desejável que as propriedades sejam mantidas.
Nominalmente, são elas:

\begin{alineas}
    \item Mundo pequeno: O diâmetro das redes complexas geradas pelo modelo deve ter uma relação logarítmica com a quantidade de vértices no modelo.
    \item distribuição de graus em lei de potência: Os graus dos vértices devem estar distribuídos com uma lei de potência.
    \item Homofilia: O grafo gerado deve apresentar uma tendência de priorização da adjacência com vértices semelhantes.
    \item Estrutura de comunidades: O grafo gerado deve de ter comunidades, conforme etiquetadas na cobertura, de forma que todo vértice pertença a uma ou mais comunidades, e as comunidades se organizem em uma estrutura hierárquica.
    \item Comunidades homogêneas: As comunidades devem ser coesas não apenas na perspectiva topológica, mas em similaridade.
\end{alineas}

\end{document}
