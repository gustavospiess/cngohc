\documentclass[notes.tex]{subfiles}

\begin{document}

No estudo de redes complexas, é relevante o entendimento de algumas terminologias que serão utilizadas mais à frente.
É definido por \citeonline{fortunato2010community} o conceito de \emph{clique}, um clique se refere à um conjunto de vértices que forma um sub grafo completo.
Definições ligeiramente distintas podem ser utilizadas, como apontado pelo próprio \citeonline{fortunato2010community}, dependendo do contexto, por exemplo é apresentada a definição de um \emph{n-clique}, como sendo um subgrafo onde todos os vértices estão a \emph{n} ou menos arestas de distância.
Ou ainda um \emph{k-clique chain}, sendo a união de dois cliques com \emph{k} ou mais vértices que compartilham um ou mais desses vértices.

Outra terminologia relevante é o conceito de partição, definido por \citeonline{fortunato2010community} como sendo a divisão dos vértices de um grafo em conjuntos distintos de forma que todos os vértices pertençam a exatamente uma partição.
Essas duas definições são relevantes na forma como elas interagem com a definição de comunidade.
Em alguns contextos uma comunidade pode ser definida como sendo necessariamente um \emph{clique} \cite{fortunato2010community}, ou em outros contextos a divisão das comunidades no grafo deve de representar uma partição \cite{da2020comparative, fortunato2010community}

Mais uma definição definição relevante pode ser encontrada no trabalho de \citeonline{fortunato2010community}.
O autor define triângulo como sendo um conjunto de três vértices conectados entre si, i.e. um \emph{3-clique}.
E a contra parte dessa definição, o coeficiente de aglomeração, definido como a proporção em que as triplas conexas de um grafo são triângulos.
Isso é, para um coeficiente de aglomeração igual a meio, isso representa que para quaisquer três vértices que estejam conectados entre si, existe cinquenta por cento de chance de esses vértices formarem um triângulo.

Por fim, no estudo de redes complexas, existe também a definição de um grafo aleatório.
Um grafo aleatório é produzido a partir de um grafo qualquer, mantendo a mesma quantidade de vértices e mantendo o mesmo grau, mas considerando que os vértices estão conetados baseados em uma probabilidade uniformemente distribuída.
Um grafo aleatório é geralmente utilizado para comparação, servindo como exemplo nulo, no sentido de que ele não apresenta características topológicas relevantes \cite{fortunato2010community}.

\end{document}
